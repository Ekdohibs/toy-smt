\documentclass[a4paper,10pt]{article}

\usepackage[francais]{babel} 
%\usepackage[T1]{fontenc}
\usepackage[utf8]{inputenc}

\usepackage{hyperref}

\usepackage{amsfonts}
\usepackage{amsmath}
\usepackage{amssymb}

%\usepackage{algorithmic}
%\usepackage{algorithm}
\usepackage{listings}
\usepackage{url}
\usepackage{graphicx}

\usepackage{listings}
\usepackage{xcolor}
%\lstset{upquote=false,
%        columns=flexible,
%        keepspaces=true,
%        breaklines,
%        breakindent=0pt}
%\lstset{basicstyle=\ttfamily\footnotesize,
%        commentstyle=\itshape\color{green},
%        keywordstyle=\bfseries\color{red},
%				stringstyle=\it\color{blue}}
%\lstset{language={Python}}
%\lstset{xleftmargin=2em,xrightmargin=2em,aboveskip=\topsep,belowskip=\topsep,
%        frame=single,rulecolor=\color{blue},backgroundcolor=\color{blue!5},frame=tb}

\textwidth 173mm \textheight 235mm \topmargin -50pt \oddsidemargin
-0.45cm \evensidemargin -0.45cm

%\newcommand{\code}[1]{#1}
\newcommand{\code}[1]{{\fontfamily{pcr}\selectfont #1}}

\title{Un petit solver SMT pour la théorie de l'égalité}
\author{Nathanaël Courant}

\lstset{language=Caml}

\begin{document}

\maketitle

\section{Utilisation}

\code{esmt fichier.cnfuf} éxécute le solver sur le fichier fourni. Le
résultat est affiché sur la sortie standard, et est soit \code{No
  solution} si le fichier d'entrée n'a aucune solution, soit une liste
de séries d'égalités, une par ligne, deux variables sur la même ligne
étant différentes.

\section{Implantation}

\subsection{Décision de la théorie de l'égalité}

La procédure de décision pour la théorie de l'égalité s'effectue en
temps amorti $O(\alpha (n) + \log (r))$ par opération, où $n$ est le
nombre de variables et $r$ le nombre de clauses de différence.

Pour cela, on utilise la structure de données suivante~:
\begin{lstlisting}
type state = Union_find.t * ISet.t Parray.t * int
\end{lstlisting}

Ici, on a une structure \textit{union-find} qui gère les égalités, un
tableau qui envoie chaque racine de l'\textit{union-find} vers les
identifiants des inégalités dont un de ses enfants est
membre, et un entier indiquant le plus petit identifiant d'inégalité
libre.

Cela permet de répondre à toutes les requêtes en le temps annoncé (et
l'utilisation de tableaux persistants rend le backtracking en
complexité $O(1)$ amorti si la version avant backtracking n'est pas
réutilisée).

\subsection{Solveur SAT}

Le solveur SAT est basé sur l'idée des \textit{two-watched literals},
avec de plus une revérification complète des littéraux qui peuvent
avoir été déduits d'égalités précédentes par la procédure de décision
de la théorie de l'égalité avant de décider une nouvelle affectation
de littéral.

Une autre version fournie dans le code, mais non utilisée, est la même
chose sans les \textit{two-watched literals}, elle semble être entre
$3$ et $5$ fois moins rapide à l'éxécution, lorsque les clauses
contiennent $3$ littéraux chacune.

\section{Tests et performance}

Cette implantation se débrouille raisonnablement bien sur des
problèmes qui ont jusqu'à de l'ordre de 25 variables et 500 clauses~:
ainsi, une solution est trouvée en environ 20 secondes~; pour 25
variables, 700 clauses, et 5 littéraux par clause, une solution est
trouvée en 2 minutes.

Pour des problèmes plus grands (30 variables, 1000 clauses), les
tableaux persistants provoquent un débordement de pile au moment du
backtracking.

Le script utilisé pour générer les tests est fourni~; un encodage de
la 3-coloriabilité du graphe de Petersen l'est également, ainsi qu'un
programme cherchant la solution du
\href{https://en.wikipedia.org/wiki/Zebra_Puzzle}{problème du zèbre}
(30 variables, 90 clauses).


\end{document}
